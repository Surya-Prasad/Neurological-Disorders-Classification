%\documentclass[PhD,two side]{ugthesis}
%\documentclass[MS]{ugthesis}
%\documentclass[MTech]{ugthesis}
\documentclass[BTech]{ugthesis}
\usepackage{times}
\usepackage{t1enc}
\usepackage{tikz}
\usepackage{subfigure}
\usepackage{pgfplots}
\usepackage{setspace} 
\usepackage{geometry}
\usepackage{graphicx}
\usepackage{epstopdf}
\usepackage{lscape}
\usepackage{fancyhdr}
\usepackage{natbib}
\usepackage{hyperref} % hyperlinks for references.
\usepackage{amsmath} % easier math formulae, align, subequations \ldots
\usepackage{amssymb}
\usepackage{wasysym}
\usepackage{titlesec}
\usepackage{textcomp}
\usepackage{pifont}
\usepackage{appendix} 
\usetikzlibrary{decorations.pathmorphing}
\usetikzlibrary{shapes,arrows,shadows,patterns}
\usepackage[printonlyused]{acronym}
%\usepackage{nomencl}
%\newcommand{\bigsize}{\fontsize{16pt}{20pt}\selectfont}
%\renewcommand\nomname{\centerline {NOTATION}}
%\makenomenclature
\setcounter{MaxMatrixCols}{20}
\captionsetup[figure]{labelfont=bf}
\begin{document}
%%%%%%%%%%%%%%%%%%%%%%%%%%%%%%%%%%%%%%%%%%%%%%%%%%%%%%%%%%%%%%%%%%%%%%
% Title page

\title{Title of project } % Enter The Project Title

\firstauthor{student1}% Enter The Student name
\firstauthorregno{[cb.en.u4cse17xxx]}
\secondauthor{ student2 }% Enter The Student name
\secondauthorregno{[cb.en.u4cse17yyy]}
\thirdauthor{ student3 } % If there is no third author, leave the space blank like \thirdauthor{}
\thirdauthorregno{[cb.en.u4cse17zzz]}
\fourthauthor{student3}
\fourthauthorregno{[cb.en.u4cse17zyx]}
\fifthauthor{ }
\fifthauthorregno{}
\guide{Dr. M. Xyz, Ph.D} % Enter your guide's name
\designation{Professor} % Enter your guide's designation
\guidedepartment{Computer Sciene \& Engineering} % Enter the department name of your Guide 
\hod{Dr. D. KINGSLY JEBA SINGH} % Enter HOD's name
\department{Computer Sciene \& Engineering} % Enter your department name
\date{Review \#1 -- September 2020} % Enter month and year of submission
%\nocite{*}

\maketitle
%%%%%%%%%%%%%%%%%%%%%%%%%%%%%%%%%%%%%%%%%%
% Abstract

\abstract
\begin{doublespacing}
{\large\noindent Type your abstract here.\\
 Abstract should be one page synopsis of the project report typed double line spacing. Just type in your abstract here.}
\end{doublespacing}

\pagebreak
%%%%%%%%%%%%%%%%%%%%%%%%%%%%%%%%%%%%%%%%%%%%%%%%%%%%%%%%%%%%%%%%%%%%%%
% Acknowledgements
%\acknowledgements
%Not applicable for interim reviews!\\



%\begin{flushright}
%%\end{flushright}
%%%%%%%%%%%%%%%%%%%%%%%%%%%%%%%%%%%%%%%%%%%%%%%%%%%%%%%%%%%%%%%%%
% Table of contents etc.

\begin{singlespace}
\tableofcontents
\thispagestyle{empty}

\listoftables
\addcontentsline{toc}{chapter}{LIST OF TABLES}
\listoffigures
\addcontentsline{toc}{chapter}{LIST OF FIGURES}
\end{singlespace}


%%%%%%%%%%%%%%%%%%%%%%%%%%%%%%%%%%%%%%%%%%%%%%%%%%%%%%%%%%%%%%%%%%%%%%
\abbreviations
%\begin{acronym}[longest acronym must be entered here]
\begin{acronym}[OKID/ERA]

%\acro{acronym}{in detail}
\acro{ABC}{Artificial Bee Colony}
\acro{ACO}{Ant Colony Optimization}
\acro{BA}{Bees Algorithm}
\acro{BFO}{Bacterial Foraging Optimization}
\acro{BM} {Bending Moment}
\acro{CMIR}{Condensed Model Identification and Recovery}
\acro{CMTM}{Consistent Mass Transfer Matrix}
\acro{CPU}{Central Processing Unit}
\acro{CS}{Cuckoo Search}
\acro{CSI}{Complete Structural Identification}
\acro{DAQ}{Data Acquisition}
\acro{DOF}{Degrees Of Freedom}
\acro{DTM}{Damped Transfer Matrix}
\acro{EA}{Evolutionary Algorithm}
\acro{EKF}{Extended Kalman Filter}
\acro{ERA}{Eigen system Realization Algorithm}
\acro{FE}{Finite Element}
\acro{FRF}{Frequency Response Function}
\acro{GA}{Genetic Algorithm}
\acro{LCTM}{Lumped Crack Transfer Matrix}
\acro{LM}{Levenberg-Marquardt}
\acro{LMTM}{Lumped Mass Transfer Matrix}
\acro{LS}{Least Square}
\acro{MAE}{Mean Absolute Error}
\acro{MSE}{Mean Square Error}
\acro{MSI}{Modular Smart Interface}
\acro{OKID/ERA}{Observer Kalman Filter Identification/Eigen Realization Algorithm}
\acro{PSO}{ Particle Swarm Optimization}
\acro{SA}{Simulated Annealing}
\acro{SCTM}{Single Crack Transfer Matrix}
\acro{SF} {Shear Force}
\acro{SHM}{Structural Health Monitoring}
\acro{SI}{Structural Identification}
\acro{SS}{Sub-Structure}
\acro{SSI}{Sub-Structural Identification}
\acro{TCTM}{Two Crack Transfer Matrix}
\acro{TM}{Transfer Matrix}
\end{acronym}
% Use the syntax \ac{acronym} whereever you use this acronym.
% Abbreviations

%\noindent 
%\begin{tabbing}
%xxxxxxxxxxx \= xxxxxxxxxxxxxxxxxxxxxxxxxxxxxxxxxxxxxxxxxxxxxxxx \kill
%\textbf{TM}   \> Transfer Matrix \\
%\textbf{LMTM} \> Lumped Mass Transfer matrix \\
%\textbf{CMTM} \> Consistent Mass Transfer matrix \\
%\textbf{SCTM} \> Single Crack Transfer matrix \\
%\textbf{LCTM} \> Lumped Crack Transfer matrix \\
%\textbf{DCTM} \> Double Crack Transfer matrix \\
%\textbf{DOF} \> Degrees Of Freedom \\
%\textbf{GA} \> Genetic Algorithm  \\
%\textbf{PSO} \> Particle Swarm Optimization \\
%\textbf{SI} \> Structural Identification \\
%\end{tabbing}

\pagebreak

%%%%%%%%%%%%%%%%%%%%%%%%%%%%%%%%%%%%%%%%%%%%%%%%%%%%%%%%%%%%%%%%%%%%%%
% Enter the symbols used in the thesis in alphabatical order
\chapter*{\centerline{LIST OF SYMBOLS}}
\addcontentsline{toc}{chapter}{LIST OF SYMBOLS}
%\nomenclature{b}{Width of the beam}
%\nomenclature{r}{Number of DOF}
%\nomenclature{n}{Number of elements}
%\nomenclature{h}{Thickness of the beam}
%\nomenclature{$\theta$}{Length of the beam}
%\nomenclature{$\omega$}{Circular frequency}
\begin{doublespace}
\begin{tabbing}
%\printnomenclature
xxxxxxxxxxx \= xxxxxxxxxxxxxxxxxxxxxxxxxxxxxxxxxxxxxxxxxxxxxxxx \kill
\textbf{$\alpha$, $\beta$}   \> Damping constants  \\
\textbf{$\theta$}   \> Angle of twist, rad  \\
\textbf{$\omega$}   \> Angular velocity, rad/s  \\
\textbf{$b$}   \> Width of the beam,  m \\
\textbf{$h$}   \> Height of the beam,  m \\
\textbf{$\{f(t)\}$}   \> force vector  \\
\textbf{$[K^e]$}  \> Element stiffness matrix\\
\textbf{$[M^e]$}  \> Element mass matrix \\
\textbf{$\{q(t)\}$}   \> Displacement vector  \\
\textbf{$\{\dot{q}(t)\}$}   \> Velocity vector  \\
\textbf{$\{\ddot{q}(t)\}$}   \> Acceleration vector  \\


\end{tabbing}
\end{doublespace}

\pagebreak
\clearpage
% The main text will follow from this point so set the page numbering
% to arabic from here on.
\pagenumbering{arabic}


%%%%%%%%%%%%%%%%%%%%%%%%%%%%%%%%%%%%%%%%%%%%%%%%%%
% Introduction.

%Enter your chapter number here
\chapter{INTRODUCTION}
\label{chap:intro}
\emph{Write your introduction to your project domain and motivation. Replace the sampe text with yours.}
\section{Problem Definition}
\emph{Define your problem here}
 \ac{SI} is typically an inverse process whereby structural parameters such as stiffness, damping properties are identified from input excitation and output responses.
 

 Generally, Engineering problems can be classified into forward and inverse problems \citep{783214}. In forward problems, \citep{two}the system output responses are calculated from the known system properties and input responses as shown in Figure \ref{fig:fp} whereas in inverse problems, the system parameters are identified based on the input and output responses of the system which is shown in Figure \ref{fig:ip}. 
 \begin{figure}[htpb]
  \centering
    \tikzstyle{block} = [rectangle, line width=0.75pt,draw,  text width=20mm, text centered,  minimum height=10mm]
    \tikzstyle{line} = [draw, line width=0.75pt, -latex']
   \begin{tikzpicture}
  \node [block,text width=30mm] (input) {\bf\footnotesize Input Excitation};
  \node [block,right of=input,text width=35mm,node distance = 45mm] (nummode) {\bf \footnotesize Numerical model};
   \node [block, above of=nummode,text width=35mm,node distance = 20mm] (para) {\bf \footnotesize System Parameters (Mass, Stiffness and Damping co-efficient)};
  \node [block,right of=nummode,text width=35mm,node distance = 45mm] (output) {\bf \footnotesize Output Responses (Acceleration, Velocity and Displacement)};
  % Draw edges
        \path [line] (input) -- (nummode);
        \path [line] (para) -- (nummode);
         \path [line] (nummode) -- (output);
 \end{tikzpicture}
 \caption{Forward problem}
 \label{fig:fp}
 \end{figure}
 \begin{figure}[htpb]
   \centering
     \tikzstyle{block} = [rectangle, line width=0.75pt,draw,  text width=20mm, text centered,  minimum height=10mm]
     \tikzstyle{line} = [draw, line width=0.75pt, -latex']
    \begin{tikzpicture}
   \node [block,text width=30mm] (input) {\bf\footnotesize Input Excitation};
   \node [block,right of=input,text width=35mm,node distance = 45mm] (nummode) {\bf \footnotesize Numerical model};
    \node [block, right of=nummode,text width=35mm,node distance = 45mm] (para) {\bf \footnotesize System Parameters (Mass, Stiffness and Damping co-efficient)};
   \node [block,above of=nummode,text width=35mm,node distance = 20mm] (output) {\bf \footnotesize Output Responses (Acceleration, Velocity and Displacement)};
   % Draw edges
         \path [line] (input) -- (nummode);
         \path [line] (output) -- (nummode);
          \path [line] (nummode) -- (para);
  \end{tikzpicture}
  \caption{Inverse problem}
  \label{fig:ip}
  \end{figure}
  For a structure, the input excitation is a periodic force and the output responses are displacement, velocity and acceleration. The input force can be measured using force transducer and the output responses can be measured respectively using vibration pick-ups, velometer and accelerometer. Some \ac{SI} algorithms require measurement of all responses or any one of the output responses. Since the input and output responses are measurable for a structure with unknown parameters, the \ac{SI} problem is an inverse problem which identifies structural or damage parameters.\\ \vspace{-5mm}
  \subsection{Sub sections} 
  Sub-sections must be numbered as shown in this text. 
  \subsubsection{Sub-sub sections}
  Sub-sections of sub section are not to be numbered and it should be in bold as shown in this text. \\
\textbf{Important}
The last paragraph of this chapter shoud be a summary of the remaining part of the report.

%------------------------------------------------------------------------------------------------% 
\chapter{LITERATURE SURVEY}

This chapter is to exlain your lit. survey
\section{Frequency Domain SI}
\citet{george} showed that the magnitude of change in natural frequencies is a function of the severity and of the location of deterioration in structures. The modal analysis has been carried out on a welded steel frame and a wire rope with damage. \citet{gredia} proposed a direct method for determining six flexural stiffnesses of a thin anisotropic plate. In this method, natural frequencies \citep{Leo} and mode shapes have been processed using \ac{LS} technique.
\section{Particle Swarm Optimization}
A basic variant of the \ac{PSO} \citep{kenndy} algorithm works by having a population (called a swarm) of candidate solutions (called particles). These particles are moved around in the search-space according to a few simple formulae. The movements of the particles are guided by their own best known position in the search-space as well as the entire swarm's best known position. When improved positions are being discovered these will then come to guide the movements of the swarm. The process is repeated and by doing so it is hoped, but not guaranteed, that a satisfactory solution will eventually be discovered.
\section{Summary}
Here a summary can be given, justifying your approach taken for this project.

\section{Data Set( If applicable)}
Here expain your dataset and explain the reason for selecting this dataset
\section{Software/Tools Requirements}
Here you can list the tools/librries/packages etc. with a brief description
%------------------------------------------------------------------------------------------------%
\chapter{Modularization \& Plan}
This chapter is for explaining your plan for completing the project.
\section{ Responsibilities of Team Members}
Here you can provide the roles and responsibiities of team members; explain for each member.
\section{Month Wise Plan}
Brief plan for each month; pls refer the slide template for details.
%%%%%%%%%%%%%%%%%%%%%%%%%%%%%%%%%%%%%%%%%%%%%%%%%%%%%%%%%%%%
% Bibliography.

\begin{singlespace}
  \bibliography{thesis_refer} % Enter your .bib file name here
\end{singlespace}
%%%%%%%%%%%%%%%%%%%%%%%%%%%%%%%%%%%%%%%%%%%%%%%%%%%%%%%%%%%%
  \end{document}
